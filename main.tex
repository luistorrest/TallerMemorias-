\documentclass{article}
\usepackage[utf8]{inputenc}
\usepackage[spanish]{babel}
\usepackage{listings}
\usepackage{graphicx}
\graphicspath{ {images/} }
\usepackage{cite}

\begin{document}

\begin{titlepage}
    \begin{center}
        \vspace*{1cm}
            
        \Huge
        \textbf{Proyecto investigativo}
        
            
        \vspace{0.5cm}
        \LARGE
        Taller de memorias
            
        \vspace{1.5cm}
            
        \textbf{Luis Fernando Torres Torres}
            
        \vfill
            
        \vspace{0.8cm}
            
        \Large
        Despartamento de Ingeniería Electrónica y Telecomunicaciones\\
        Universidad de Antioquia\\
        Medellín\\
        Septiembre de 2020
            
    \end{center}
\end{titlepage}

\tableofcontents

\newpage

\section{Introduccion}\label{intro}
Uno de los componentes más importantes dentro de la computación es la memoria, la cual esta estrechamente relacionada con el concepto de almacenamiento de información que sera procesada posteriormente.En este documento se introduce de manera clara y sencilla, ciertos conceptos que son de vital importancia para entender la definición, el funcionamiento y tipos de memoria que usa el computador para realizar cualquier proceso o tarea.

\section{Que es la memoria del computador.} \label{contenido}

\section{Tipos de memoria.} \label{contenido}%Explicar usos de cada memoria
\subsection{Memoria RAM}
\subsection{Memoria ROM}
\subsection{Disco Duro}
\subsection{Memoria cache}


\section{Como se gestiona la memoria en un computador.} \label{contenido}

\section{¿Qué hace que una memoria sea más rápida que otra? ¿Por qué esto es importante?}.\label{contenido}

\section{Conclusiones}.\label{contenido}


\bibliographystyle{IEEEtran}
\bibliography{references}

\end{document}
